\documentclass{article}
\usepackage{doi}
\usepackage{url}
\usepackage{cite}

\begin{document}


A delta wing body,
denoted ``Model 4'' in a 1973 wind tunnel report%
\cite{wing-planform-boom-wt-tn-7160}


Two groups of researchers computed the signature at
at 0.3 body lengths and propagated it to 3.6 body lengths.
Cliff and Thomas\cite{cliff-thomas-boom}
used a structured grid of 1.5 million nodes and
an unstructured grid of 177 thousand nodes.
Djomehri and Erickson\cite{felisa-adapt-boom}
used an adapted grid with 193 thousand nodes.
Madson\cite{madison-tranir-boom} used 
an adaptive Cartesian full-potential
method to 0.1, 0.2, and 0.325 body lengths with
a grid of 330 thousand boxes,
which was then propagated to 3.6 body lengths.
Kandil et al.\cite{kandil-ssbe} apply a 
structured Euler code to 0.4 body lengths,
which is further propagated to 3.6 body lengths
with a full potential code for the delta wing case at $C_L=0.08$.
Cheung, Edwards, and Lawrence\cite{cheung-edwards-lawrence-cfd-boom-extrap} 
applied various methods
to examine the same geometry at 2.7 Mach 
with signals propagated to 3.1 body lengths.
None of the previously reported methods used a single
Euler method to a distance of 3.6 body lengths.

\ref{madison-tranir-boom}

check INRIA?

Castner\ref{castner-plume-delta-wing-boom}
\ref{castner-lake-plume-delta-boom}

cliff\ref{cliff-thomas-nasa-tm-2008-unstruct-euler-refinement-rotation-delta}



\bibliography{../references}
\bibliographystyle{../std-aiaa}

\end{document}
